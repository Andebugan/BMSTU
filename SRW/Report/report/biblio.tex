\begin{thebibliography}{3}
\bibitem{f2}
Робот Ф-2 [Электронный ресурс] – 2021. – Режим доступа: http://f2robot.com/robot/ (дата обращения: 29.11.2021)
\bibitem{lang}
Кибрик А. Е. Язык // Языкознание. Большой энциклопедический словарь. — Большая Российская энциклопедия, 1998. — С. 605.
\bibitem{petty}
Педди, Джон. История визуальной магии в компьютерах : как создаются красивые изображения в САПР, 3D, VR и AR. - Лондон: Спрингер, 2013. - C. 25.
\bibitem{rogers}
Роджерс Д., Адамс Дж. Математические основы машинной графики. - Москва, Мир, 2001. - С. 604.
\bibitem{belcg} %http://e.biblio.bru.by/bitstream/handle/1212121212/6479/178_Komputernay_grafika.pdf?sequence=1&isAllowed=y
Шилов А. В., Лесковец И. В. Компьютерная графика - Могилев, Государственное учреждение высшего профессионального образования «Белорусско-Российский университет», 2014 - С. 38.
\bibitem{cgshish}
Шишкин Е. В., Боресков А. В., Зайцев А. А. - Москва, "ДИАЛОГ-МИФИ", 1993. - С. 135.
\bibitem{cgshish_2}
Шишкин Е. В., Боресков А. В. - Москва, "ДИАЛОГ-МИФИ", 1996. - С. 290.
\end{thebibliography}