\chapter*{Введение}
\addcontentsline{toc}{chapter}{Введение}
Ф-2 --- это аффективный робот, разрабатываемый командой российских исследователей, обладающий умением общаться с людьми. С его помощью можно разрабатывать стратегии диалога, мимику и жесты, изменения направления взгляда и многое другое \cite{robot_2}. Робот сделан максимально простым, чтобы его можно было легко собрать. Использование данной платформы позволяет исследовать эмоциональный контакт, возникающий за счёт поведения робота, а не за счёт его внешнего вида [Volkova L., Ignatev A., Kotov N., Kotov A. 2021, 163-176]. Робот напоминает мультипликационных персонажей, которые совсем не похожи на человека, но при этом эмоциональны и симпатичны благодаря своим жестам и мимике, а не из-за внешнего сходства с человеком. Робот реагирует на слова: он принимает на вход текст на естественном языке, строит смысл этого текста, выбирает коммуникативную цель (многие из которых - выразить эмоцию) и выполняет мультимодальную реакцию (она включает речь, жесты и/или мимику), характерные для неё. Когнитивный компонент робота отвечает за мышление и за понимание текста. Для входящего высказывания когнитивный компонент должен построить некоторый набор умозаключений, выбрать эмоции, которые может вызывать текст, чтобы далее проявить эти эмоции в поведении робота посредством воспроизведения срисованных с человека эмоциональных реакций. Робот может читать книги в виде текстовых файлов, новости и блоги через подписку RSS, воспринимать устную речь через сторонний сервис преобразования в письменную форму. По письменному тексту для каждого предложения когнитивный компонент строит синтаксическое дерево, а затем конструирует семантическое представление --- смысл текста. Смысл текста вызывает у робота различные умозаключения, выводы и ответные реакции. Эти процессы моделируются системой отношений типа «если-то» --- сценариев. Робот сравнивает смысл поступившего текста с посылками всех сценариев и активизирует ближайшие сценарии. Смысл поступившего текста может активизировать сценарии, ответственные за эмоциональную обработку --- ''Меня никто не любит'', ''Я никому не нужен'' или, наоборот, ''Приятно быть в центре внимания''. Когнитивный компонент может работать отдельно от робота: он будет прочитывать множество текстов, конструировать их смысл, приписывать каждому смыслу аффективную реакцию и сохранять результаты. При управлении роботом каждый сценарий может сформировать поведенческий пакет на языке BML и передать его роботу для выполнения. Таким образом, поведение робота составляется из BML-пакетов --- реакций на окружающие события или просто движений в состоянии покоя: когда роботу нечего делать, он будет слегка двигаться \cite{f2} 

Одной из проблем данного робота является невозможность проводить эксперименты без наличия физической копии робота. Решить данную проблему представляется возможным при помощи создания виртуальной среды, которая позволит эмулировать робота Ф-2 и набор входных данных (положение головы, ограниченная мимика и жестикуляция).\\

Создание подобной виртуальной среды будет иметь следующие применения в проекте Ф-2.
\begin{enumerate}
	\item Будет предоставлена возможность воспроизведения реакций робота в тестовом режиме на этапе разработки элементов реакции или набора реакций, воспроизводимых совместно.
	\item Станет возможной постановка экспериментов по взаимодействию робота с респондентами в виртуальном режиме. Это позволит проводить больше экспериментов по коммуникации и, следовательно, получать больше обратной связи от респондентов, в том числе позволит респондентам взаимодействовать с роботом удалённо вместо того или вместе с тем, чтобы приезжать к роботу.
	\item Станет возможным проводить параллельные эксперименты по коммуникации человека и робота, что позволит объединить усилия множества респондентов, что приведет к более интенсивному развитию проекта на материале собранной обратной связи.
\end{enumerate}

Целью данной работы является обзор существующих решений для задачи создания трёхмерной среды для визуализации окружения робота-собеседника Ф-2. Для достижения поставленной цели необходимо выполнить следующие задачи:
\begin{itemize}
\item рассмотреть существующие решения для создания трёхмерной виртуальной среды;
\item провести анализ рассматриваемых решений;
\item сформировать критерии для сравнения существующих решений;
\item провести сравнительный анализ проанализированных решений по сформированным критериям.
\end{itemize}
