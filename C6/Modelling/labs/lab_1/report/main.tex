\chapter{Цели и задачи работы}
\textbf{Цель работы} --- получение навыков решения задачи Коши для ОДУ методами Пикара и
явными методами первого порядка точности (Эйлера) и второго порядка точности (Рунге-Кутта).

\chapter{Исходные данные}
ОДУ, не имеющее аналитического решения:\\
$u'(x) = x^{2} + u^{2}$\\
$u(0) = 0$

\chapter{Результаты работы}
1. Таблица, содержащая значения аргумента с заданным шагом в интервале [0, xmax] и результаты расчета функции u(x) в приближениях Пикара (от 1-го до 4-го), а также численными методами. Границу интервала xmax выбирать максимально возможной из условия, чтобы численные методы обеспечивали точность вычисления решения уравнения u(x) до второго знака после запятой.
2. График функции в диапазоне [-xmax, xmax].

\chapter{Выполнение}
\section{Рассчёт $x_{max}$}
Для рассчёта $x_{max}$ для численных методов использовалось правило Рунге, заключающееся в том, что точность численных методов на i-ом шаге примерно равна $L = \frac{|y_{i, h} - y_{i, h/2}|}{2^{p} - 1}$, где $h$ - шаг, $p$ - степень точности. Для явного метода Эйлера она равна 1, для метода Рунге-Кутта, использующегося в данной рабораторной она равна 2.

\chapter{Контрольные вопросы}
\begin{enumerate}
\item В работе использовался явный метод Эйлера, для того, чтобы применять неявный метод. В явном методе Эйлера - $y_{n+1} = y_{n} + hf(x_{n}, y_{n})$, в неявном методе Эйлера - $y_{n+1} = y_{n} + hf(x_{n+1}, y_{n+1})$.  В нашем случае получается $y_{n+1} = y_{n} + h (x_{n+1}^{2} + y_{n+1}^{2})$, то есть $y_{n+1} = y_{n} + h x_{n+1}^{2} + h y_{n+1}^{2}$. В итоге мы имеем квадратное уравнение $y_{n+1}^{2} - \frac{1}{h} y_{n+1} + \frac{1}{h} y_{n} + x_{n+1}^{2} = 0$ относительно $y_{n+1}$ (остальные величины известны). $D = (\frac{1}{h})^{2} - 4 \cdot (h x_{n+1}^{2} + \frac{1}{h} y_{n})$, корни будут $y_{(n+1)_{1}} = \frac{-\frac{1}{2} - \sqrt{(\frac{1}{h})^{2} - 4 \cdot (h x_{n+1}^{2} + \frac{1}{h} y_{n})}}{2}, y_{(n+1)_{2}} = \frac{-\frac{1}{2} + \sqrt{(\frac{1}{h})^{2} - 4 \cdot (h x_{n+1}^{2} + \frac{1}{h} y_{n})}}{2}$. Выбираем меньший корень, иначе решение уйдёт в плюс бесконечность.
\end{enumerate}