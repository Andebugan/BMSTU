\chapter{Цели и задачи работы}
\textbf{Цель работы} --- получение навыков решения задачи Коши для ОДУ методами Пикара и
явными методами первого порядка точности (Эйлера) и второго порядка точности (Рунге-Кутта).

\chapter{Исходные данные}
ОДУ, не имеющее аналитического решения:
$u'(x) = x^{2} + u^{2}$\\
$u(0) = 0$

\chapter{Результаты работы}
1. Таблица, содержащая значения аргумента с заданным шагом в интервале [0, xmax] и результаты расчета функции u(x) в приближениях Пикара (от 1-го до 4-го), а также численными методами. Границу интервала xmax выбирать максимально возможной из условия, чтобы численные методы обеспечивали точность вычисления решения уравнения u(x) до второго знака после запятой.
2. График функции в диапазоне [-xmax, xmax].

\chapter{Выполнение}
\section{Рассчёт $x_{max}$}
Для рассчёта $x_{max}$ для численных методов использовалось правило Рунге, заключающееся в том, что точность численных методов на i-ом шаге примерно равна $L = \frac{|y_{i, h} - y_{i, h/2}|}{2^{p} - 1}$, где $h$ - шаг, $p$ - степень точности. Для явного метода Эйлера она равна 1, для метода Рунге-Кутта, использующегося в данной рабораторной она равна 2.



\chapter{Контрольные вопросы}