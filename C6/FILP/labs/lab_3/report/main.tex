\chapter{Цели и задачи работы}
\textbf{Цель работы} --- приобрести навыки работы в системе Common Lisp.
\textbf{Задачи работы} --- изучить работу форм — функций, которые особым образом обрабатывают
свои аргументы и особенности их работы в Lisp.

\chapter{Теоретические вопросы}
\section{Базис Lisp}
Базис Lisp составляют атомы, структуры, базовые функции и встроенные и спецальные функционалы.
\section{Классификация функций}
Функция - однозначное отображение множества значений аргументов в значение функции.\\
Функциональный язык - тот, который базируется на понятии функции.\\
Функциональность системы - предоставляемые пользователю возможности.\\

Классификация функций по аргументам и поведению.
\begin{itemize}
\item чистые функции (математические функции) - имеют фиксированное число аргументов, для определённого набора аргументов один фиксированный результат
\item формы (специальные функции) - функции, принимающие на вход произвольное количество аргументов, или по-разному обрабатывающее аргументы.
\item псевдо-функции - функции, обладающие побочным эффектом. Побочный эффект - событие, изменяющее сознание системы. Пример - setf, связывающее атом и значение и format, выводящее значение на экран.\item функционалы - принимают функции в качестве параметров либо в качестве возвращаемого значения выступает функция.
\end{itemize}

Классификация функций по именованию.
\begin{itemize}
\item именованные - есть имя, определяется через defun. Специальные символы (T, Nil) и самоопределимые атомы (натуральные, вещественные числа, строки) не могут выступать в роли функции.
\item неименованные - нет имени, через lambda.
\end{itemize}

\section{Способы создания функций}
Создание именованной функции - синтаксис:\\
(defun имя список\_аргументов лямбда-выражение)\\

Создание неименованной функции - синтаксис:\\
(lambda список\_аргументов лямбда-выражение)\\
(lambda $(x_{1}, ..., x_{k})$ форма)\\

\section{Функции cond, if, and/or}
Функция cond - (cond (test\_1 body\_1) (test\_2 body\_2) ... (test\_i body\_i) [(t body\_i+1)]), test\_i - тестирующееся выражение, body\_i - выполняется, если test\_i корректный.\\
Функция if - (if test than else)/(if test than) - если нет else, то автоматически
возращается nil.\\
Логические функции and/or (and test\_1 test\_2 ... test\_n) - поочереди вычисляет test\_1 --- test\_n, если одно из них nil, то возвращает nil, иначе возвращает последнее вычисленное значение. (or test\_1 test\_2 ... test\_n) - поочереди вычисляет test\_1 --- test\_n, возвращает первое вычисленное значение которое не является nil.
\chapter{Практические задания}
\section{Задание 1}
Написать функцию, которая принимает целое число и возвращает первое
четное число, не меньшее аргумента.
\begin{lstlisting}
(defun even_greater_and_equal (x)
  (if (= (logand x 1) 0)
      x
      (+ x 1)))
\end{lstlisting}
\section{Задание 2}
Написать функцию, которая принимает число и возвращает число
того же знака, но с модулем на 1 больше модуля аргумента.
\begin{lstlisting}
(defun abs_inc (x)
  (if (>= x 0)
      (+ x 1)
      (- x 1)))
\end{lstlisting}
\section{Задание 3}
Написать функцию, которая принимает два числа и возвращает
список из этих чисел, расположенный по возрастанию.\\

\begin{lstlisting}
(defun mysort (a b)
  (if (>= a b)
      (list b a)
      (list a b)))
\end{lstlisting}
\section{Задание 4}
Написать функцию, которая принимает три числа и возвращает Т только
тогда, когда первое число расположено между вторым и третьим.\\

\begin{lstlisting}
(defun func (a b c)
  (if (and (> b a) (> c b))
      t
  ))
\end{lstlisting}
\section{Задание 5}
Каков результат вычисления следующих выражений?\\

(and 'fee 'fie 'foe) $\rightarrow$ 'foe\\
(or 'fee 'fie 'foe) $\rightarrow$ 'fee\\
(or nil 'fie 'foe) $\rightarrow$ 'fie\\
(and nil 'fie 'foe) $\rightarrow$ nil\\
(and (equal 'abc 'abc) 'yes) $\rightarrow$ 'yes\\
(or (equal 'abc 'abc) 'yes) $\rightarrow$ t\\

\section{Задание 6}
Написать предикат, который принимает два числа-аргумента и возвращает
Т, если первое число не меньше второго.\\

\begin{lstlisting}
(defun pred (a b)
  (>= a b))
\end{lstlisting}
\section{Задание 7}
Какой из следующих двух вариантов предиката ошибочен и почему?\\

(defun pred1 (x) \\
(and (numberp x) (plusp x)))\\

(defun pred2 (x) \\
(and (plusp x)(numberp x)))\\

Ошибочен второй предикат, так как если x - не число, то возникнет ошибка, поскольку проверка на число проходит после plusp.

\section{Задание 8}
Решить задачу 4, используя для ее решения конструкции
IF, COND, AND/OR.\\

С if:
\begin{lstlisting}
(defun func (a b c)
  (if (and (> b a) (> c b))
      t
  ))
\end{lstlisting}

С cond:
\begin{lstlisting}
(defun func (a b c)
  (cond ((and (> b a) (> c b) t))))
\end{lstlisting}

\section{Задание 9}Переписать функцию how-alike, приведенную в лекции и использующую COND, используя
только конструкции IF, AND/OR.\\

Реализация с cond
\begin{lstlisting}
(defun how-alike (x y)
  (cond
    ((or (= x y) (equal x y)) 'the_same)
    ((and (oddp x) (oddp y)) 'both_odd)
    ((and (evenp x) (evenp y)) 'both_even)
    (T 'difference)))
\end{lstlisting}

Реализация с if
\begin{lstlisting}
(defun how-alike (x y)
  (if (if (= x y)
          (equal x y))
      'the_same
      (if (if (oddp x)
              (oddp y))
          'both_odd
          (if (if (evenp x)
                  (evenp y))
              'both_even
              'difference))))
\end{lstlisting}

Реализация с and/or
\begin{lstlisting}
(defun how-alike (x y)
  (or (and (or (= x y) (equal x y)) 'the_same)
  (and (oddp x) (oddp y) 'both_odd)
  (and (evenp x) (evenp y) 'both_even)
  'difference))
\end{lstlisting}