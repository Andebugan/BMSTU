\chapter{Цели и задачи работы}
\textbf{Цель работы} --- познакомиться со средой Visual Prolog, познакомиться со структурой программы: способом запуска и формой вывода результатов.
\textbf{Задачи работы} --- изучить принципы работы в среде VisualProlog, возможность получения однократного и многократного результата, изучить базовые конструкции языка Prolog, структуру програмым Prolog, форму ввода исходных данных и вывода результатов работы программы.

\chapter{Теоретические вопросы}
Что собой представляет программа на Prolog - программа представляет из себя базу знаний, состоящую из аксиом (или фактов) и теорем, которые задаются с помощью терм, являющихся логическими переменными.

Структура программы на Prolog - Программа на Prolog состоит из разделов. Каждый раздел начинается со своего заголовка. Структура программы:\\
Директивы компилятора рективы компилятора —— зарезервизарезервирроованные символьные константыванные символьные константы\\
CONSTANTS —— раздел описания константраздел описания констант\\
DOMAINS —— раздел описания доменовраздел описания доменов\\
DATABASE —— раздел описания предикатов внутренней базы данныхраздел описания предикатов внутренней базы данных\\
PREDICATES —— раздел описания предикатовраздел описания предикатов\\
CLAUSES —— раздел описания предложений базы знанийраздел описания предложений базы знаний\\
GOAL —— раздел описания внутренней цели (вопроса).раздел описания внутренней цели (вопроса).\\
В программе не обязательно должны быть все разделы.\\

Каковы результаты работы программы - в качестве результата Prolog возвращает ответ на вопрос в виде ''да'' или ''нет'' и если да, то также все значения переменных, которые позволили получить ответ ''да'', если переменные использовались в вопросе.\\

\chapter{Практические задания}
\section{Задание 1}
Разработать программу - "Телефонный справочник". Протестировать работу программы.
\begin{lstlisting}
predicates
nondeterm phone(symbol, symbol)

clauses
phone("Pavel", "8(916)521-23-16").
phone("Kirill", "8(915)522-74-51").
phone("Alexander", "8(914)511-03-52").
phone("Pavel", "8(913)311-65-45").
phone("Maxim", "8(914)641-49-14").
phone("Vladimir", "8(916)821-15-73").

goal
phone(X,Y).
\end{lstlisting}

Тестирование программы:\\
\begin{lstlisting}
phone("Pavel", "0").
no

phone("Pavel", "8(915)522-74-51"). 
yes

phone("Pavel", X).
X=8(916)521-23-16
X=8(913)311-65-45
2 Solutions	

phone(X, "8(914)511-03-52").
X=Alexander
1 Solution  

phone(X, Y).
X=Pavel, Y=8(916)521-23-16
X=Kirill, Y=8(915)522-74-51
X=Alexander, Y=8(914)511-03-52
X=Pavel, Y=8(913)311-65-45
X=Maxim, Y=8(914)641-49-14
X=Vladimir, Y=8(916)821-15-73
6 Solutions
\end{lstlisting}

\section{Задание 2}
Составить программу базу знаний , с помощью которой можно определить, например, множество студентов, обучающихся в одном ВУЗе и их телефоны. Студент может одновременно обучаться в нескольких ВУЗах. Привести примеры возможных вариантов вопросов и варианты ответов (не менее 3 х). Описать порядок формирования вариантов ответа. Исходную базу знаний сформировать с помощью только фактов.\\
*Исходную базу знаний сформировать, используя правила.\\
** Разработать свою базу знаний (содержание произвольно)\\

\begin{lstlisting}
domains
name=symbol
phone_number=string
surname=symbol
lastname=symbol
university=symbol

predicates
nondeterm phone(name, surname, lastname, phone_number)
nondeterm student(name, surname, lastname, university)
nondeterm person(name, surname, lastname, university, phone_number)

clauses
phone(peter, ivanov, andreevich, "8(219)-892-22-70").
phone(konstantin, knyazev, bespalovich, "8(779)-991-46-67"). 
phone(german, trofimov, semenovich, "8(441)-495-69-57").     
phone(kirill, avdeev, osipovich, "8(882)-634-60-37").        
phone(andrew, ivanov, grigorievich, "8(453)-375-89-10").     
phone(pavel, ivanov, pavlovich, "8(377)-586-23-67").
phone(konstantin, yakushev, samoilovich, "8(495)-735-20-15").
phone(pavel, knyazev, bespalovich, "8(787)-104-49-39").      
phone(german, odintsov, bespalovich, "8(333)-845-58-82").    
phone(alexander, ivanov, pavlovich, "8(353)-566-30-43").
   
student(peter, ivanov, andreevich, bmstu).
student(konstantin, knyazev, bespalovich, mipt).
student(german, trofimov, semenovich, itmo).
student(kirill, avdeev, osipovich, mipt).
student(andrew, ivanov, grigorievich, tsu).
student(pavel, ivanov, pavlovich, itmo).
student(konstantin, yakushev, samoilovich, nsu).
student(pavel, knyazev, bespalovich, mepi).
student(german, odintsov, bespalovich, bmstu).
student(alexander, ivanov, pavlovich, nsu).

person(Name, Surname, Lastname, University, Phone):-student(Name, Surname, Lastname, University),phone(Name, Surname, Lastname, Phone).
goal	
person(Name, Surname, Lastname, bmstu, Phone).
\end{lslisting}

Примеры возможных варинатов вопросов и варинаты ответов:
\begin{lstlisting}
person(Name, Surname, Lastname, bmstu, Phone).

Name=peter, Surname=ivanov, Lastname=andreevich, Phone=8(219)-892-22-70
Name=german, Surname=odintsov, Lastname=bespalovich, Phone=8(333)-845-58-82
2 Solutions


person(Name, ivanov, Lastname, University, Phone).

Name=peter, Lastname=andreevich, University=bmstu, Phone=8(219)-892-22-70
Name=andrew, Lastname=grigorievich, University=tsu, Phone=8(453)-375-89-10
Name=pavel, Lastname=pavlovich, University=itmo, Phone=8(377)-586-23-67
Name=alexander, Lastname=pavlovich, University=nsu, Phone=8(353)-566-30-43
4 Solutions


person(Name, Surname, Lastname, University, "8(441)-495-69-57").

Name=german, Surname=trofimov, Lastname=semenovich, University=itmo
1 Solution
\end{lstlisting}