\chapter{Цели и задачи работы}
\textbf{Цель работы} --- получить навыки построения модели предметной области, разработки и оформления программы на Prolog, изучить принципы, логику  разработки и оформления программы на Prolog.
\textbf{Задачи работы} --- приобрести навыки декларативного описания предметной области с использованием фактов и правил. Изучить способы использования термов, переменных, фактов и правил в программе на Prolog, принципы и правила сопоставления и отождествления, порядок унификации.

\chapter{Ответы на вопросы}
(Часть 1)\\
Что такое терм?\\
Единственная синтаксическая конструкция языка prolog, которая может быть либо константой, либо переменной, либо составным термом.

Что такое предикат в матлогике?\\
Предикат в математической логике это символ, представляющий свойство или отношение.

Что описывает предикат в Prolog?\\
Предикат в prolog описывает отношение между термами.

Назовите виды предложений в программе и приведите примеры таких предложений из вашей программы. Какие предложения являются основными, а какие - не основными? Каковы: синтаксис и семантика (формальный смысл) этих предложений (основных и неосновных)?\\
Предложения - могут быть основными и неосновными. Основные предложения не содержат переменных. Пример синтаксиса:\\
\begin{lstlisting}
Основное:
a(1, 2, 3).
b(4, 2, 3).
Неосновное:
c(A, B, C) := a(A, B, C), b(A, B, C).
\end{lstlisting}

Каковы назначение, виды и особенности использования переменных в программе на Prolog? Какое предложение БЗ сформулировано в более общей - абстрактной форме: содержащее или не содержащее переменных?\\
Назначение переменных - передача информации во времени и пространстве. Виды переменных - неанонимные и анонимные. Неанонимные не конкретизируются в отличие от анонимных. Предложение БЗ, содержащее переменные сформулировано в более абстрактной форме.

Что такое подстановка?\\
Это такое множество пар $\{X_i = t_i\}$, где $t_i$ - терм, не содержащий переменных (в худшем случае не содержащий $X_i$). То есть $t_i$ является значением для $X_i$.

Что такое пример терма? Как и когда строится? Как вы думаете, система строит и хранит примеры?\\
Терм B называется примером терма A, если существует $\Theta$ такое, что $B = A\Theta$, где $\Theta$ - множество подстановок. Пример терма строится во время доказательства, когда система ищет значения для терма, позволяющие ответить на поставленный вопрос ''да''. Да, система строит и хранит примеры в процессе нахождения ответа на вопрос.

(Часть 2)\\
В какой части правила сформулировано знание? Это знание о чём, с формальной точки зрения?\\
Знание формулируется в заголовке правила. Это знание об отношении между термами, входящими в правило.

Что такое процедура?\\
Это независимая именованная часть программы, которую после однократного описания можно многократно вызвать по имени из последующих частей программы для выполнения определенных действий.

Сколько в БЗ текущего задания процедур?\\
Две.

Что такое пример терма, это частный случай терма, пример? Как строится пример?\\
Терм B называется примером терма A, если существует $\Theta$ такое, что $B = A\Theta$, где $\Theta$ - множество подстановок. Пример терма строится во время доказательства, когда система ищет значения для терма, позволяющие ответить на поставленный вопрос ''да''.\\

Пример примера:\\
Пусть есть терм a(1, 2, 3) и b(A, B, C). Тогда b будет является примером a при A = 1, B = 2 и C = 3.\\

Построение примеров - пример строится на основе сравнения термов. Всего есть 4 вида термов - константа, неосновной или основной символьный терм, составной терм. Пусть у нас есть термы A и B. Тогда B является примером A, если:\\
1) B - несоставной терм, A - несоставной терм и B = A\\
2) B - составной терм, A - составной терм и арности A и B совпадают, совпадают значения главных функторов и термы-аргументы A и B попарно совпадают. Если A или B, содержит переменные, то производится их конкретизация для того, чтобы определить правильность попарного совпадения.

Что такое наиболее общий пример?\\
Наиболее общий пример это такое пример, которые является более общим по отношению к любому другому примеру.

Назначение и результат работы алгоритма унификации. Что значит двунаправленная передача параметров при работе алгоритма унинфикации, поясните на примере одно из случаев пункта 3.\\
Назначение алгоритма унификации - проверка термоф на унифицируемости. Результат работы - 0 (неуспех) или 1 (успех). Двунаправленная передача параметров при работе алгоритма унификации - 

В каком случае запускается механизм отката?\\
Если решение не найдено и невозможен переход в новое состояние (тупик).

Виды и назначение переменных в Prolog. Примеры из задания. Почему использованы те или другие переменные (примеры из задания)?\\
Назначение переменных - передача информации во времени и пространстве. Виды переменных - неанонимные и анонимные. Неанонимные не конкретизируются в отличие от анонимных.

\chapter{Практические задания}
Составить программу т.е. модель предметной области базу знаний , объединив в ней информацию знания
\begin{itemize}
\item «Телефонный справочник»: Фамилия, №тел, Адрес структура (Город, Улица, №дома, №кв)
\item «Автомобили»: Фамилия владельца, Марка, Цвет, Стоимость, и др.,Марка, Цвет, Стоимость, и др.
\item «Вкладчики банков»: Фамилия, Банк, счет, сумма, др.Фамилия, Банк, счет, сумма, др.
\end{itemize}
Владелец может иметь несколько телефонов, автомобилей, вкладов (Факты).

Используя правила, обеспечить возможность поиска:\\
1.\\
а) По № телефона найти: Фамилию, Марку автомобиля, Стоимость автомобиля милию, Марку автомобиля, Стоимость автомобиля (может быть несколько)\\
б) Используя сформированное а) правило, по № телефона найти: только Марку автомобиля (автомобилей может быть несколько)\\
2. Используя простой, не составной вопрос: по Фамилии (уникальна в городе, но в разных городах есть однофамильцы) и Городу проживания найти: Улицу проживания,, Банки, в которых есть вклады и №телефона.\\

Для задания1 и задания2:\\
для одного из вариантов ответов, и для а) и для б), описать словесно порядок поиска ответа на вопрос, указав, как выбираются знания, и, при этом, для каждого этапа унификации, выписать подстановку наибольший общий унификатор, и соответствующие примеры термов.\\

Используя конъюнктивное правило и простой вопрос , обеспечить возможность поиска:\\
По Марке и Цвету автомобиля найти Фамилию, Город, Телефон и Банки, в которых владелец автомобиля имеет вклады. Лишней информации не находить и не передавать!!! Владельцев может быть несколько  - не более 3-х, один и ни одного.\\
1. Для каждого из трех вариантов словесно подробно описать порядок формирования ответа (в виде таблицы). При у казать отметить моменты очередного запуска алгоритма унификации и полный результат его работы. Обосновать следующий шаг работы системы. Выписать унификаторы подстановки. Указать моменты, причины и результат отката, если он есть.\\
2. Для случая нескольких владельцев (2-х) приведите примеры работы системы при разных порядках следования в БЗ процедур и знаний в них «Телефонный справочник», «Автомобили», «Вкладчик и банков», или «Автомобили», «Вкладчики банков», «Телефонный справочник» )). Сделайте вывод:\\ Одинаковы ли множество работ и объем работ в разных случаях\\
3. Оформите 2 таблицы, демонстрирующие порядок работы алгоритма унификации вопроса и подходящего заголовк а правила (для двух слуаев из пункта 2) и укажите результаты его работы: ответ и побочный эффект.\\

\begin{lstlisting}
domains
phone_number=string
surname=symbol

city=symbol
street=symbol
house_num=integer
flat_num=integer

mark=symbol
color=symbol
price=integer
bank_name=symbol
dep_num=integer
dep_size=integer

adress=adress(city, street, house_num, flat_num)

predicates
nondeterm phone(surname, phone_number, adress)
nondeterm car(surname, mark, color, price)
nondeterm bank(surname,bank_name,dep_num,dep_size)
nondeterm find_car(phone_number,surname,mark,price)
nondeterm find_mark(phone_number,mark)
nondeterm find_phone_street_bank(surname, phone_number, city, street, bank_name)
nondeterm find_owner(color, mark, surname, phone_number, city, bank_name)

clauses
phone(trofimov, "8(972)-236-88-62",adress(moscow,krasnaya,89,199)).
phone(trofimov, "8(118)-902-79-19",adress(moscow,krasnaya,89,199)).
phone(ivanov, "8(481)-716-34-95",adress(tula,imperskaya,37,179)).
phone(ivanov, "8(100)-890-60-39",adress(tula,imperskaya,37,179)).
phone(osipov, "8(541)-954-80-76",adress(samara,petrovstaya,81,184)).
phone(osipov, "8(172)-769-16-96",adress(samara,petrovstaya,81,184)).
phone(smirnov, "8(415)-724-38-49",adress(st_petersburg,kerenski_bulvar,28,192)).
phone(smirnov, "8(533)-971-59-87",adress(st_petersburg,kerenski_bulvar,28,192)).
phone(avdeev, "8(294)-313-58-75",adress(st_petersburg,imperskaya,27,118)).      
phone(avdeev, "8(716)-742-39-78",adress(st_petersburg,imperskaya,27,118)).      
phone(lebedev, "8(691)-645-30-38",adress(moscow,krasnaya,66,174)).
phone(avdeev, "8(372)-485-47-49",adress(moscow,krasnaya,76,176)).

bank(trofimov,tinkoff,8546,1252536).
bank(trofimov,alphabank,2029,2521060).
bank(ivanov,vtb,8864,1299690).
bank(ivanov,sberbank,5744,3933822).
bank(osipov,raffaizen,7899,4160502).
bank(osipov,tinkoff,7970,2237517).
bank(smirnov,vtb,9641,1012682).
bank(smirnov,sberbank,8605,4974929).
bank(avdeev,vtb,9197,1050565).
bank(avdeev,raffaizen,7828,4905040).
bank(lebedev,alphabank,9270,846207).
bank(avdeev,tinkoff,1505,3401347).

car(trofimov,volkswagen,red,1600000).
car(trofimov,volkswagen,blue,1600000).
car(ivanov,audi,red,1600000).
car(ivanov,bmw,yellow,1450000).
car(osipov,honda,grey,1300000).
car(osipov,honda,blue,1300000).
car(smirnov,bmw,green,1450000).
car(smirnov,volkswagen,black,1600000).
car(avdeev,volkswagen,green,1600000).
car(avdeev,honda,red,1300000).
car(lebedev,volkswagen,blue,1600000).
car(avdeev,hyundai,black,1470000).

find_car(Phone, Surname, Mark, Price) :- car(Surname, Mark, _, Price), phone(Surname, Phone, _).
find_mark(Phone, Mark) :- find_car(Phone, _, Mark, _).
find_phone_street_bank(Surname, Phone, City, Street, Bank) :- phone(Surname, Phone, adress(City, Street, _, _)), bank(Surname, Bank, _, _).
find_owner(Color, Mark, Surname, Phone, City, Bank) :- phone(Surname, Phone, adress(City, _, _, _)), bank(Surname, Bank, _, _), car(Surname, Mark, Color, _).
goal	
%find_car("8(972)-236-88-62", Surname, Mark, Price).
%find_mark("8(972)-236-88-62", Mark).
%find_phone_street_bank(trofimov, Phone, City, Street, Bank).
find_owner(red, audi, Surname, Phone, City, Bank).

\end{lstlisitng}

Примеры вариантов вопросов и ответов:
\begin{lstlisting}
Вопрос:
find_car("8(972)-236-88-62", Surname, Mark, Price).

Результат
Surname=trofimov, Mark=volkswagen, Price=1600000
Surname=trofimov, Mark=volkswagen, Price=1600000
2 Solutions

Вопрос:
find_mark("8(972)-236-88-62", Mark).

Результат:
Mark=volkswagen
Mark=volkswagen
2 Solutions

Вопрос:
find_phone_street_bank(trofimov, Phone, City, Street, Bank).

Результат:
Phone=8(972)-236-88-62, City=moscow, Street=krasnaya, Bank=tinkoff
Phone=8(972)-236-88-62, City=moscow, Street=krasnaya, Bank=alphabank
Phone=8(118)-902-79-19, City=moscow, Street=krasnaya, Bank=tinkoff
Phone=8(118)-902-79-19, City=moscow, Street=krasnaya, Bank=alphabank
4 Solutions
\end{lstlisting}