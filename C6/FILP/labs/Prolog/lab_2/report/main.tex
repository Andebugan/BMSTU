\chapter{Цели и задачи работы}
\textbf{Цель работы} --- получить навыки построения модели предметной области, разработки и оформления программы на Prolog, изучить принципы, логику  разработки и оформления программы на Prolog.
\textbf{Задачи работы} --- приобрести навыки декларативного описания предметной области с использованием фактов и правил. Изучить способы использования термов, переменных, фактов и правил в программе на Prolog, принципы и правила сопоставления и отождествления, порядок унификации.

\chapter{Ответы на вопросы}
(Часть 1)\\
Что такое терм?\\

Что такое предикат в матлогике?\\

Что описывает предикат в Prolog?\\

Назовите виды предложений в программе и приведите примеры таких предложений из вашей программы. Какие предложения являются основными, а какие - не основными? Каковы: синтаксис и семантика (формальный смысл) этих предложений (основных и неосновных)?\\

Каковы назначение, виды и особенности использования переменных в программе на Prolog? Какое предложение БЗ сформулировано в более общей - абстрактной форме: содержащее или не содержащее переменных?\\

Что такое подстановка?\\

Что такое пример терма? Как и когда строится? Как вы думаете, система строит и хранит примеры?\\

(Часть 2)\\
В какой части правила сформулировано знание? Это знание о чём, с формальной точки зрения?\\

Что такое процедура?\\

Сколько в БЗ текущего задания процедур?\\

Что такое пример терма, это частный случай терма, пример? Как строится пример?\\

Что такое наиболее общий пример?\\

Назначение и результат работы алгоритма унификации. Что значит двунаправленная передача параметров при работе алгоритма унинфикации, поясните на примере одно из случаев пункта 3.\\

В каком случае запускается механизм отката?\\

Виды и назначение переменных в Prolog. Примеры из задания. Почему использованы те или другие переменные (примеры из задания)?\\

\chapter{Практические задания}
\section{Задание 1}
Разработать программу - "Телефонный справочник". Протестировать работу программы.
\begin{lstlisting}
predicates
nondeterm phone(symbol, symbol)

clauses
phone("Pavel", "8(916)521-23-16").
phone("Kirill", "8(915)522-74-51").
phone("Alexander", "8(914)511-03-52").
phone("Pavel", "8(913)311-65-45").
phone("Maxim", "8(914)641-49-14").
phone("Vladimir", "8(916)821-15-73").

goal
phone(X,Y).
\end{lstlisting}

Тестирование программы:\\
\begin{lstlisting}
phone("Pavel", "0").
no

phone("Pavel", "8(915)522-74-51"). 
yes

phone("Pavel", X).
X=8(916)521-23-16
X=8(913)311-65-45
2 Solutions	

phone(X, "8(914)511-03-52").
X=Alexander
1 Solution  

phone(X, Y).
X=Pavel, Y=8(916)521-23-16
X=Kirill, Y=8(915)522-74-51
X=Alexander, Y=8(914)511-03-52
X=Pavel, Y=8(913)311-65-45
X=Maxim, Y=8(914)641-49-14
X=Vladimir, Y=8(916)821-15-73
6 Solutions
\end{lstlisting}

\section{Задание 2}
Составить программу базу знаний , с помощью которой можно определить, например, множество студентов, обучающихся в одном ВУЗе и их телефоны. Студент может одновременно обучаться в нескольких ВУЗах. Привести примеры возможных вариантов вопросов и варианты ответов (не менее 3 х). Описать порядок формирования вариантов ответа.\\
Исходную базу знаний сформировать с помощью только фактов.\\
*Исходную базу знаний сформировать, используя правила.\\
** Разработать свою базу знаний (содержание произвольно)\\

\begin{lstlisting}
domains
name=symbol
phone_number=string
surname=symbol
lastname=symbol
university=symbol

predicates
nondeterm phone(name, surname, lastname, phone_number)
nondeterm student(name, surname, lastname, university)
nondeterm person(name, surname, lastname, university, phone_number)

clauses
phone(peter, ivanov, andreevich, "8(219)-892-22-70").
phone(konstantin, knyazev, bespalovich, "8(779)-991-46-67"). 
phone(german, trofimov, semenovich, "8(441)-495-69-57").     
phone(kirill, avdeev, osipovich, "8(882)-634-60-37").        
phone(andrew, ivanov, grigorievich, "8(453)-375-89-10").     
phone(pavel, ivanov, pavlovich, "8(377)-586-23-67").
phone(konstantin, yakushev, samoilovich, "8(495)-735-20-15").
phone(pavel, knyazev, bespalovich, "8(787)-104-49-39").      
phone(german, odintsov, bespalovich, "8(333)-845-58-82").    
phone(alexander, ivanov, pavlovich, "8(353)-566-30-43").
   
student(peter, ivanov, andreevich, bmstu).
student(konstantin, knyazev, bespalovich, mipt).
student(german, trofimov, semenovich, itmo).
student(kirill, avdeev, osipovich, mipt).
student(andrew, ivanov, grigorievich, tsu).
student(pavel, ivanov, pavlovich, itmo).
student(konstantin, yakushev, samoilovich, nsu).
student(pavel, knyazev, bespalovich, mepi).
student(german, odintsov, bespalovich, bmstu).
student(alexander, ivanov, pavlovich, nsu).

person(Name, Surname, Lastname, University, Phone):-student(Name, Surname, Lastname, University),phone(Name, Surname, Lastname, Phone).
goal	
person(Name, Surname, Lastname, bmstu, Phone).
\end{lstlisting}

Примеры возможных варинатов вопросов и варинаты ответов:
\begin{lstlisting}
person(Name, Surname, Lastname, bmstu, Phone).

Name=peter, Surname=ivanov, Lastname=andreevich, Phone=8(219)-892-22-70
Name=german, Surname=odintsov, Lastname=bespalovich, Phone=8(333)-845-58-82
2 Solutions


person(Name, ivanov, Lastname, University, Phone).

Name=peter, Lastname=andreevich, University=bmstu, Phone=8(219)-892-22-70
Name=andrew, Lastname=grigorievich, University=tsu, Phone=8(453)-375-89-10
Name=pavel, Lastname=pavlovich, University=itmo, Phone=8(377)-586-23-67
Name=alexander, Lastname=pavlovich, University=nsu, Phone=8(353)-566-30-43
4 Solutions


person(Name, Surname, Lastname, University, "8(441)-495-69-57").

Name=german, Surname=trofimov, Lastname=semenovich, University=itmo
1 Solution
\end{lstlisting}