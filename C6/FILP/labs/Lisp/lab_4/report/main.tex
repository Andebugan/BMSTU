\chapter{Цели и задачи работы}
\textbf{Цель работы} --- приобрести навыки работы с управляющими структурами Lisp.
\textbf{Задачи работы} ---изучить работу функций с произвольным количеством аргументов, функций
разрушающих и неразрушающих структуру исходных аргументов.

\chapter{Теоретические вопросы}
\section{Синтаксическая форма и хранение программы в памяти}
В LISP формы представления программы и обрабатываемых ею данных одинаковы и представляются в виде S-выражений. Поэтому программы могут обрабатывать и преобразовывать другие программы и даже самих себя. В процессе трансляции можно введенное и сформированное в результате вычислений выражение данных проинтерпретировать в качестве программы и непосредственно выполнить. Так как программа представляет собой S-выражение, в памяти она представлена либо как атом (5 указателей; форма представления атома в памяти), либо списковой ячейкой (бинарный узел; 2 указателя).

\section{Трактовка элементов списка}
Первый аргумент списка, который поступает на вход интерпретатору, трактуется как имя функции, остальные --- как аргументы этой функции.
\section{Порядок реализации программы}
Программа в языке LISP представляется S-выражением, которое передается интерпретатору --- функции eval, которая выводит последний, полученный после обработки S-выражения, результат.

\section{Способы определения функции}
Определение именованной функции - синтаксис:\\
(defun имя список\_аргументов лямбда-выражение)\\

Определение неименованной функции - синтаксис:\\
(lambda список\_аргументов лямбда-выражение)\\
(lambda $(x_{1}, ..., x_{k})$ форма)\\

\chapter{Практические задания}
\section{Задание 1}
Чем принципиально отличаются функции cons, list, append?\\
Принципиальное отличие cons list и append состоит в способе работы со списками. cons и list создают новые списки, изменяя ссылки, в то время как append работает с копией списка, что позволяет не разрушать структуру списка, поданного на вход.\\

Пусть:
(setf lst1 '(a b))\\
(setf lst2 '(c d))\\

Каковы результаты вычисления следующих выражений?\\
(cons lstl lst2) $\rightarrow$ ((a b) c d)\\
(list lst1 lst2) $\rightarrow$ ((a b) (c d))\\
(append lst1 lst2) $\rightarrow$ (a b c d)\\

\section{Задание 2}
Каковы результаты вычисления следующих выражений, и почему?\\
(reverse ()) $\rightarrow$ nil\\
(last ())  $\rightarrow$ nil\\
(reverse '(a))  $\rightarrow$ (a)\\
(last '(a)) $\rightarrow$ (a)\\
(reverse '((a b c))) $\rightarrow$ ((a b c))\\
(last '((a b c))) $\rightarrow$ ((a b c))\\

\section{Задание 3}
Написать, по крайней мере, два варианта функции, которая возвращает последний элемент
своего списка-аргумента.

\begin{lstlisting}
(defun last_elem (x)
  (if (cdr x)
      (last_elem (cdr x))
      (car x)))
\end{lstlisting}

\begin{lstlisting}
(defun last_elem (x)
  (car (reverse x)))
\end{lstlisting}

\section{Задание 4}
Написать, по крайней мере, два варианта функции, которая возвращает свой список-
аргумент без последнего элемента.

\begin{lstlisting}
(defun remove_last (lst)
  (and lst (nreverse (cdr (reverse lst)))))
\end{lstlisting}

\begin{lstlisting}
(defun delete_last_rec (lst acc)
                 (if (not (eql (cdr lst) nil))
                     (delete_last_rec (cdr lst) (cons (car lst) acc))
                     (reverse acc)))

(defun delete_last (lst)
  (delete_last_rec lst nil))
\end{lstlisting}

\section{Задание 5}
Написать простой вариант игры в кости, в котором бросаются две правильные кости. Если
сумма выпавших очков равна 7 или 11 -- выигрыш, если выпало (1,1) или (6,6) --- игрок право
снова бросить кости, во всех остальных случаях ход переходит ко второму игроку, но
запоминается сумма выпавших очков. Если второй игрок не выигрывает абсолютно, то
выигрывает тот игрок, у которого больше очков. Результат игры и значения выпавших костей
выводить на экран с помощью функции print.

\begin{lstlisting}
(defun throw_bones ()
  (if  (setf *random-state* (make-random-state t))
       (list (random 7) (random 7))
       ))

(defun sum_score (x)
  (+ (car x) (cadr x)))

(defun player_1_throw (scores)
  (if (setf score (throw_bones))
      (and (setf scores
                    (list
                     (sum_score score)
                     (cadr scores)))
           (or (format t  "~&~&Player 1 throws~&results:~&~A ~A~&scores:~&Player 1: ~A~&Player 2: ~A~&" (car score) (cadr score) (car scores) (cadr scores)) t)
           (if (or (eql (car scores) 11)
                   (eql (car scores) 7))
               (format t  "~&>Player 2 wins<")
               (if (or
                    (and
                     (eql (car score) 1)
                     (eql (cadr score) 1)
                     )
                    (and
                     (eql (car score) 6)
                     (eql (cadr score) 6)))
                   (or (format t  "~&Player 1 can get another try ~&Throw again? [y/n] ")
                        (if (eql (read) 'y)
                            (player_1_throw scores)
                            (player_2_throw scores)))
                   (player_2_throw scores)
                   ))
           )))

(defun player_2_throw (scores)
  (if (setf score (throw_bones))
      (and (setf scores
                 (list
                  (car scores)
                  (sum_score score)
                  ))
           (or (format t  "~&~&Player 2 throws~&results:~&~A ~A~&scores:~&Player 1: ~A~&Player 2: ~A~&" (car score) (cadr score) (car scores) (cadr scores)) t)
           (if (or (eql (cadr scores) 11)
                   (eql (cadr scores) 7))
               (format t  "~&>Player 2 wins<")
               (if (or
                    (and
                     (eql (car score) 1)
                     (eql (cadr score) 1)
                     )
                    (and
                     (eql (car score) 6)
                     (eql (cadr score) 6)))
                   (or (format t  "~&Player 2 can get another try ~&Throw again? [y/n] ")
                       (if (eql (read) 'y)
                           (player_2_throw scores)
                           (player_2_throw scores)))
                   (if (>= (car scores) (cadr scores))
                       (if (eql (car scores) (cadr scores))
                           (format t  "~&Draw")
                           (format t  "~&Player 1 wins"))
                       (format t  "~&Player 2 wins"))
                   ))
           )))

(defun play ()
  (player_1_throw '(0 0)))
\end{lstlisting}