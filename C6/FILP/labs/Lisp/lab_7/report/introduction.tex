\Introduction
Графом $G(V, E)$ называется совокупность двух множеств - непустого множества $V$ и множества $E$ неупорядоченных пар различных элементов множества $V$. Множество $V$ называется множеством вершин, множество $E$ называется множеством ребер \cite{but}. Граф $G$ называется полным, если в его состав входит хотя бы одна вершина, и граф $G$ называется связным, если любые две его вершины соединены путём в G \cite{distel}.

Целью данного рубежного контроля является разработка параллельного алгоритма поиска в глубину в неориентированном графе из заданной вершины. Для того, чтобы достичь поставленной цели необходимо выполнить следующие задачи:

\begin{enumerate}
	\item провести анализ алгоритма поиска в глубину;
	\item провести анализ параллельного алгоритма поиска в глубину;
	\item описать используемые в алгоритмах структуры данных;
	\item привести схемы рассматриваемых алгоритмов;
	\item программно реализовать описанные выше алгоритмы;
	\item провести сравнительный анализ скорости работы алгоритмов по времени относительно количества вершин в графе.
\end{enumerate}