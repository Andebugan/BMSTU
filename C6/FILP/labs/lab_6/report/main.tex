\chapter{Цели и задачи работы}
\textbf{Цель работы} --- приобрести навыки использования функционалов.
\textbf{Задачи работы} --- изучить работу и методы использования применяющих и отображающих функционалов: apply, funcall, mapcar, maplist.

\chapter{Практические задания}
Используя функционалы решите описанные ниже задачи.
\section{Задание 1}
Напишите функцию, которая уменьшает на 10 все числа из списка-аргумента этой функции.

\begin{lstlisting}
(defun add_10 (x)
  (+ x 10)
  )

(defun add_10_lst (x)
  (mapcar 'add_10 x)
  )
\end{lstlisting}

\section{Задание 2}
Напишите функцию, которая умножает на заданное число-аргумент все числа из заданного списка-аргумента, когда:\\
a) все элементы списка - числа,\\
б) элементы списка - любые объекты.\\

Функция, если все элементы - числа:\\
\begin{lstlisting}
(defun func_mult_x (n)
  #'(lambda (x) ( * x n))
  )

(defun mult_numbers (lst x)
  (mapcar (func_mult_x x) lst)
  )
\end{lstlisting}

Функция, если элементы - любые объекты:\\
\begin{lstlisting}
(defun func_mult_check_x (n)
  #'(lambda (x)
      (if (numberp x)
          ( * x n)
          x
          )
      )
  )

(defun mult_check_numbers (lst x)
  (mapcar (func_mult_check_x x) lst)
  )
\end{lstlisting}


\section{Задание 3}
Написать функцию, которая по своему списку-аргументу lst определяет является ли он палиндромом (то есть равны ли lst и (reverse lst)).

\begin{lstlisting}
(defun check_palindrom (lst)
  (reduce #'(lambda (x y) (and x y)) (mapcar #'(lambda (x y) (eql x y)) lst (reverse lst)))
  )
\end{lstlistiing}

\section{Задание 4}
Написать предикат set-equal, который возвращает t, если два его множества-аргумента содержат одни и те же элементы, порядок которых не имеет значения.

\begin{lstlisting}
(defun set-equal (set_1 set_2)
  (eval `(and ,@(mapcar #'(lambda (x)
                     (eval `(or ,@x))
                     )
                 (mapcar #'(lambda (x)
                             (mapcar #'(lambda (y)
                                         (eql x y)
                                         )
                                     set_2
                                     )
                             )
                         set_1
                         )
                 ))
        )
  )
\end{lstlisting}

\section{Задание 5}
Написать функцию которая получает как аргумент список чисел, а возвращает список квадратов этих чисел в том же порядке.

\begin{lstlisting}
(defun square_lst (lst)
  (mapcar (lambda (x) (* x x)) lst)
  )
\end{lstlisting}

\section{Задание 6}
Напишите функцию, select-between, которая из списка-аргумента, содержащего только числа, выбирает только те, которые расположены между двумя указанными границами-аргументами и возвращает их в виде списка.

\begin{lstlisting}
(defun select-between (lst a b)
  (mapcan #'(lambda (x) (if (and (> x a) (< x b)) (list x))) lst)
  )
\end{lstlisting}

\section{Задание 7}
Написать функцию, вычисляющую декартово произведение двух своих списков-аргументов. ( Напомним, что А х В это множество всевозможных пар (a b), где а принадлежит А, b принадлежит В.)

\begin{lstlisting}
(defun decart (lst1 lst2)
  (mapcan #'(lambda (x) (mapcar #'(lambda (y) (list x y)) lst2)) lst1)
  )
\end{lstlisting}

\section{Задание 8}
Почему так реализовано reduce, в чем причина?
(reduce \#'+ 0) -> 0 - не работает, второй аргумент reduce должен быть последовательностью\\
(reduce \#'+ ()) -> 0 - так как (+) -> 0, аналогично (reduce \#'* ()) -> 1 так как (*) -> 1\\

\section{Задание 9}
Пусть list-of-list список, состоящий из списков. Написать функцию, которая вычисляет сумму длин всех элементов list-of-list, т.е. например для аргумента ((1 2) (3 4)) -> 4.

\begin{lstlisting}
(defun lst-sub-len (lst)
  (reduce #'+ (mapcar 'length lst))
  )
\end{lstlisting}