\chapter{Цели и задачи работы}
\textbf{Цель работы} --- приобрести навыки создания и использования функций пользователя в Lisp.
\textbf{Задачи работы} --- изучить работу интерпретатора Lisp, алгоритм работы функции eval, структуру и порядок обработки программы в Lisp.

\chapter{Теоретические вопросы}
\section{Базис Lisp}
Базис Lisp составляют атомы, структуры, базовые функции и встроенные и спецальные функционалы.
\section{Классификация функций}
Функция - однозначное отображение множества значений аргументов в значение функции.\\
Функциональный язык - тот, который базируется на понятии функции.\\
Функциональность системы - предоставляемые пользователю возможности.\\

Классификация функций по аргументам и поведению.
\begin{itemize}
\item чистые функции (математические функции) - имеют фиксированное число аргументов, для определённого набора аргументов один фиксированный результат
\item формы (специальные функции) - функции, принимающие на вход произвольное количество аргументов, или по-разному обрабатывающее аргументы.
\item псевдо-функции - функции, обладающие побочным эффектом. Побочный эффект - событие, изменяющее сознание системы. Пример - setf, связывающее атом и значение и format, выводящее значение на экран.\item функционалы - принимают функции в качестве параметров либо в качестве возвращаемого значения выступает функция.
\end{itemize}

Классификация функций по именованию.
\begin{itemize}
\item именованные - есть имя, определяется через defun. Специальные символы (T, Nil) и самоопределимые атомы (натуральные, вещественные числа, строки) не могут выступать в роли функции.
\item неименованные - нет имени, через lambda.
\end{itemize}

\section{Способы создания функций}
Создание именованной функции - синтаксис:\\
(defun имя список\_аргументов лямбда-выражение)\\

Создание неименованной функции - синтаксис:\\
(lambda список\_аргументов лямбда-выражение)\\
(lambda $(x_{1}, ..., x_{k})$ форма)\\

\section{Функции car и cdr}
car и cdr - базовые функции доступа к данным, хранящихся в формате списка.\\
car - принимает точечную пару или список в качестве аргумента и возвращает первый элемент (голову, значение по car-указателю) или nil, если на вход был подан пустой список.\\

cdr - принимает точечную пару или список и возвращает хвост (значение по cdr-указателю). Если список, поданный на вход непустой, то возвращается список из всех элементов, кроме первого. Если пустой, возвращается Nil.

\section{Назначение и отличия в работе cons и list}
cons - создаёт списковую ячейку и ставит указатели на два аргумента, таким образом объединяя свои аргументы в точечную пару. list - создаёт список из значений поданных на вход аргументов, причём количество аргументов может быть произвольным.\\

\chapter{Практические задания}
\section{Задание 1}
Составить диаграмму вычисления следующих выражений:\\
\newpage
\section{Задание 2}
Написать функцию, вычисляющую гипотенузу прямоугольного
треугольника по заданным катетам и составить диаграмму её вычисления.\\
(defun hypot (a b) (sqrt (+ (* a a) (* b b))))
\newpage
\section{Задание 3}
Написать функцию, вычисляющую объем параллелепипеда по 3-м его сторонам, и
составить диаграмму ее вычисления.\\
(defun volume (a b c) (* a b c))
\newpage
\section{Задание 4}
Каковы результаты вычисления следующих выражений?(объяснить возможную ошибку и
варианты ее устранения)\\

(list 'a c) $\rightarrow$ ошибка, атом c не связан ни с каким значением\\
(cons 'a (b c)) $\rightarrow$ ошибка, список не передаётся через quote, и атомы b, c не связаны ни с каким значением\\
(cons 'a '(b c)) $\rightarrow$ (a b c)\\
(caddy (1 2 3 4 5) $\rightarrow$ ошибка, некорректный вызов функции, атом caddy не связан ни с каким лямбда-выражением\\
(cons 'a  'b 'c) $\rightarrow$ ошибка, cons принимает на вход только два аргумента\\
(list 'a (b c)) $\rightarrow$  ошибка, список не передаётся через quote, и атомы b, c не связаны ни с каким значением\\
(list a '(b c)) $\rightarrow$ ошибка, атом a не связан ни с каким значением\\
(list (+  1 '(length '(1 2 3)))) $\rightarrow$ ошибка, выражение '(length '(1 2 3)) не является числом\\
\section{Задание 5}
Написать функцию longer\_then от двух списков-аргументов, которая возвращает Т, если
первый аргумент имеет большую длину.\\

(defun longer\_then (a b) (> (length a) (length b)))
\section{Задание 6}
Каковы результаты вычисления следующих выражений?\\

(cons 3 (list 5 6)) $\rightarrow$ (3 5 6)\\
(cons 3 '(list 5 6)) $\rightarrow$ (3 list 5 6)\\
(list 3 'from 9 'lives (- 9 3)) $\rightarrow$ (list 3 'from 9 'lives (- 9 3))\\
(+ (length for 2 too)) (car '(21 22 23))) $\rightarrow$ ошибка, атомы for и too не связаны ни с какими значениями\\
(cdr '(cons is short for ans)) $\rightarrow$ (cdr '(cons is short for ans))\\
(car (list one two)) $\rightarrow$ ошибка, атомы one и too не связаны ни с какими значениями\\
(car (list 'one 'two)) $\rightarrow$ one\\
\section{Задание 7}
Дана функция (defun mystery (x) (list (second x) (first x))).
Какие результаты вычисления следующих выражений?\\

(mystery (one two)) $\rightarrow$ ошибка, атомы one и two не связаны ни с какими значениями\\
(mystery one 'two)) $\rightarrow$ ошибка, функция принимает на вход только один аргумент\\
(mystery (last one two)) $\rightarrow$ ошика, на вход функции last должен подаваться список\\
(mystery free) $\rightarrow$ ошибка, атом free не связан ни с каким значением\\
\section{Задание 8}
Написать функцию, которая переводит температуру в системе Фаренгейта
температуру по Цельсию (defum f-to-c (temp)…).
Формулы: c = 5/9*(f-320); f= 9/5*c+32.0.
Как бы назывался роман Р.Брэдбери "+451 по Фаренгейту" в системе по Цельсию?\\

(defun f-to-c (temp) (* (/ 5 9) (- temp 32.0)))\\
"+451 по Фаренгейту" $\rightarrow$ "+232.77779 градус по Цельсию"
\section{Задание 9}
Что получится при вычисления каждого из выражений?
(list 'cons t NIL) $\rightarrow$ (cons t Nil)\\
(eval (list 'cons t NIL)) $\rightarrow$ (t)\\
(eval (eval (list 'cons t NIL))) $\rightarrow$ ошибка, t - не функция\\
(apply \#cons ''(t NIL)) $\rightarrow$ ошибка, возможно имелось ввиду (apply \#'cons '(t NIL)), в этом случае (t)\\
(eval NIL) $\rightarrow$ Nil\\
(list 'eval NIL) $\rightarrow$ (eval Nil)\\
(eval (list 'eval NIL)) $\rightarrow$ Nil\\