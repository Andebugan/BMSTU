\title{Лекции по экологии}
\chapter{Лекция 4.5}
(опоздал, успел только на вторую половину)
\section{Абиотический факторы среды}
Абиотический фактор среды - фактор неживой природы. Окружающие факторы, которые не зависят от человека (климат, погода, гравитация и тд...).
\subsection{Солнечная радиация}
Компоненты солнечной радиации - ультрафиолетовое, видимое и инфракрасное излучение. Видимое - используется растениями для выработки кислорода и для работы органов зрения. Инфракрасные лучи - переносят тепло. Ультрафиолетовые лучи - в случае людей используются для выработки витамина D, черезмерное облучение может привести к ускоренному старению и ожогам кожного покрова. 

Фотопериод - длина светового дня, определяет время года. Около 12 часов - на экваторе, чем дальше - тем меньше. 

Самый важный процесс, происходящий благодаря солнечному свету - фотосинтез, процесс выработки кислорода растениями.
$CO_{2} + H_{2}O \rightarrow C_{2}H_{12}O_{6} + O_{2}$

\subsection{Температура}
Связана с солнечным излучением и геотермальными источниками. Тепло на Земле определяется углом падения солнечных лучей. В летний период возможно температурная инверсия - теплые атмосферные слои опускаются ближе к поверхности. Для живых организмов оптимальная температура - от 0 до 25 градусов. Организмы могут регулировать температуру, эктотермы - от внешних истоников, экзотермы - от химических реакций внутри организма.

Эффективная температура - температура, оптимальная для жизни определённых видов животных и растений. Организмы могут адаптироваться к высоким и низким температурам. При адаптации к низком температурам, животные могут либо входит в состояние анабиоза, либо активно вырабатывать тепло, чтобы не допустить замерзание организма.

Правило Бертмана - чем быстрее животное и чем компактнее его тело, тем легче ему поддерживать температуру.

Правило Альмана - в более холодных областях у животных короче выступающие части тела.

Как для эктотермов, так и для эндотермов характерно наличие оптимальной температуры.

\subsection{Вода}
Вода выступает в качестве важнейшего экологического фактора. Работает в качестве компонента организма и среды обитания. Работает в виде натурального теплового аккумулятора, накапливая тепло летом и отдавая его зимой. Влажность и осадки являются факторами, влияющими на экологические условия. 

Все способы адаптации организма к содержанию воды в окружающей среде связаны с большим количеством процессов, компонентом которых является вода, протекающих внутри организма. Вода является основой протоплазмы и большей части жидкостей в организме. Ипользуется в капиллярной системе для транспортировки крови по телу. Испарение воды позволяет охладить организм. 

\subsection{Атмосферные газы и атмосферное давление}

Атмосфера - газовая оболочка, окружающая планету. В состав атмосферы входят азот, кислород, углекислый газ и другие газы. В атмосфере также содержатся газы техногенного происхождения. Атмосфера работает в качестве буфера между Землей и космосом. Большая часть опасных излучений задерживается атмосферой. 

Озоновый слой - слой озона, поглощающий смертельные для живых организмов лучи, приходящие из космоса.

Атмосферное давление - давление атмосферы. Чем дальше от поверхности, тем меньше давление. Давление атмофсеры может колебаться в зависимости от массы воздушных масс. Воздух обеспечивает работу слухового аппарата, работающего за счёт восприятия колебаний воздуха. Атмофсера является защитой от космических объектов, сгорающих в ней до попадания на поверхность. Атмосфера является основой климата. Атмосфера является основным источником азота, явлющегося однм из самых важных элементов во многих биологических процессов. Циклон - воздушный вихрь, вращающийся вокруг вертикальной оси. Диаметр циклона может достигать нескольких тысяч километров.