\chapter*{Введение}
\addcontentsline{toc}{chapter}{Введение}
В современном мире компьютерная графика используется во многих сферах человеческой деятельности, включающих в себя моделирование, компьютерные игры и системы автоматизированного проектирования [1]. Одной из основных областей применения компьютерной графики является построение изображений трёхмерных объектов. Данная ветвь машинной графики, или 3D-графика, широко используется в различных индустриях, начиная с киноиндустрии и заканчивая разработкой игр и симуляторов.

Теория распознавания образа — раздел информатики и смежных дисциплин, развивающий основы и методы классификации и идентификации предметов, явлений, процессов, сигналов, ситуаций и объектов, которые характеризуются конечным набором некоторых свойств и признаков. Такие задачи решаются довольно часто, например, при переходе или проезде улицы по сигналам светофора. Распознавание цвета загоревшейся лампы светофора и знание правил дорожного движения позволяет принять правильное решение о том, можно или нельзя переходить улицу. Необходимость в таком распознавании возникает в самых разных областях — от военного дела и систем безопасности до оцифровки аналоговых сигналов.

На сегодняшний день проблема распознавания образа приобрела выдающееся значение в условиях информационных перегрузок, когда человек не справляется с линейно-последовательным пониманием поступающих к нему сообщений, в результате чего его мозг переключается на режим одновременности восприятия и мышления, которому свойственно такое распознавание.

Целью данной работы является создание программного обеспечения, которое позволило бы загружать трёхмерную модель, редактировать масштаб, вращение и положение трёхмерной модели, выводить на экран и сохранять изображение трёхмерной модели, и  соответствие объекта, изображённого на двумерном снимке, с одним из объектов из списка тел. 

Трёхмерная модель загружается из объектного файла (.obj) в пространство сцены. Модель должна быть задана полигонально, поскольку таким образом снимок подели будет содержать достаточно данных для работы алгоритма нахождения подходящей модели из списка. Пространство сцены представляет из себя трёхмерное пространство, в котором находятся объекты сцены, включающие в себя виртуальную камеру, источники освещения и трёхмерный объект. Виртуальная камера представляет из себя наблюдателя в пространстве сцены, позволяющего получать двумерную проекцию сцены, которая затем отображается на экране.

Для достижения поставленной цели необходимо решить следующие задачи:
\begin{itemize}
	\item исследовать подходы к синтезу изображений;
	\item исследовать способы создания двумерного снимка;
	\item исследовать алгоритмы нахождения трёхмерного объекта по двумерному снимку;
	\item описать используемые при разработке ПО алгоритмы;
	\item определить средства программной реализации;
	\item реализовать алгоритмы отрисовки сцены;
	\item реализовать алгоритм позволяющий определять наиболее вероятное соответствие объекта, изображённого на двумерном снимке, с одним из объектов из списка тел;
	\item исследовать зависимость скорости алгоритма отрисовки от количества полигонов в объекте;
	\item исследовать зависимость скорости алгоритма отрисовки от размера проекции полигона;
	\item исследовать эффективность работы алгоритма алгоритма определения наиболее вероятного соответствия объекта на изображении объекту из списка тел.
\end{itemize}