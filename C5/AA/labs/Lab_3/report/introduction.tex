\Introduction
Алгоритм сортировки - это алгоритм, позволяющий упорядочить данные, заданные в форме массива, по некоему заранее известному правилу. Если элемент в массиве имеет множество полей, то выделяется поле или группа полей, служащее критерием упорядоченности массива. Такоие поле/поля называются ключом сортировки. Данные алгоритмы широко используются в совремнном программировании для упорядовачивания совершенно различных видов данных. Видов сортировок существует множество, однако в этой лабораторной мы будем рассматривать три конкретных алгоритма: алгоритм сортировки ''пузырьком'', алгоритм сортировки ''расчёской'', алгоритм сортировки вставками.

Целью данной лабораторной работы является изучение описанных выше алгоритмов сортировок и получение практических навыков при реализации данных алгоритмов. Для того, чтобы достичь поставленной цели, нам необходимо выполнить следующие задачи:

\begin{enumerate}
\item Провести анализ данных алгоритмов сортировки:
		\begin{enumerate} 
			\item Алгоритм сортировки ''пузырьком'';
			\item Алгоритм сортировки ''расчёской'';
			\item Алгоритм сортировки вставками;
		\end{enumerate}
	\item описать используемые структуры данных;
	\item привести схемы рассматриваемых алгоритмов;
	\item вычислить трудоёмкость для указанных выше алгоритмов;
	\item программно реализовать данные выше алгоритмы;
	\item провести сравнительный анализ каждого алгоритма по затрачиваемому в процессе работы времени.
\end{enumerate}
