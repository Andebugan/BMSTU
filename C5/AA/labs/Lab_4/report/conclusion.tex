\Conclusion
В процессе выполнения данной лабораторной работы были изучены алгоритм свёртки и параллельный алгоритм всёртки. Были выполнены анализ алгоритмов и представлены схемы алгоритмов, а также функциональная схема ПО. После чего эти алгоритмы были реализованы при помощи языка C++ в IDE Visual Studio 2019. Помимо этого были произведены эксперименты с целью получить информацию о временной производительности алгоритмов.В результате эксперимента было получено, что на квадратных матрицах размерами от 100 на 100 до 2000 на 2000 распаралеливание позволяет добиться практически шестикратного (5,931) ускорения при использовании 64 потоков. Также можно заметить, что при увеличении количества потоков в два раза, скорость выполнения уменьшается в среднем в два раза относительно предыдущей. В результате можно сделать вывод о том, что использование параллельного программированние позволяет значительно ускорить работу некоторых алгоритмов. Целью данной лабораторной работы являлось изучение параллельного программирования на примере алгоритма свёртки, что было успешно достигнуто.