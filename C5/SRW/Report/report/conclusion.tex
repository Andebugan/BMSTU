\Conclusion
В процессе выполнения данной лабораторной работы были изучены классический алгоритм умножения матриц, алгоритм Винограда для умножения матриц и оптимизированный алгоритм Винограда для умножения матриц. Были выполнены анализ алгоритмов и произведены вычисления трудоёмкости для каждого алгоритма. После чего эти алгоритмы были реализованы при помощи языка Python в IDE Visual Studio Code. Помимо этого были произведены эксперименты с целью получить информацию о временной производительности алгоритмов. В результате было получено, что на квадратных матрицы размерности от 10 до 80 алгоритмы работают в среднем одинаково. Однако после того, как размерность переходит 100, самым быстрым алгоритмом становится алгоритм Винограда оптимизированный, после него идёт классический алгоритм умножения матриц, и самым медленным становится алгоритм Винограда. В результате можно сделать вывод, что для матриц, размерностью больше 100 предпочтительно использовать оптимизированный алгоритм Винограда. Целью данной лабораторной работы являлось изучения алгоритмов умножения матриц, что было успешно достигнуто.