\chapter*{Заключение}
\addcontentsline{toc}{chapter}{Заключение}
Цель данной работы достигнута: проведён обзор существующих решений для задачи создания трёхмерной среды для визуализации окружения робота-собеседника Ф-2. Были выполнены следующие задачи:
\begin{itemize}
\item рассмотрены существующие решения для создания трёхмерной виртуальной среды;
\item проведён анализ рассматриваемых решений;
\item сформированы критерии для сравнения существующих решений;
\item проведён сравнительный анализ проанализированных решений по сформированным критериям.
\end{itemize}

На основе полученного в результате анализа сравнения существующих решений сделаны выводы о том, что в случае, когда в трёхмерной среде необходимо реализовать специфические функции на низком уровне, наилучшим решением будет использование графического API Vulkan, поскольку данный API является кроссплатформенным, распространяется бесплатно, использует C++ как основной язык и не имеет виртуальной среды выполнения. Если необходимо разработать трёхмерную виртуальную среду, затрачивая наименьшее количество человеко-часов, то лучшим решением будет использование Unity, поскольку данный игровой движок является кроссплатформенным, распространяется бесплатно в случае, если прибыль проекта не превышает установленного порога, и использует C\#, являющийся языком программирования более высокого уровня.