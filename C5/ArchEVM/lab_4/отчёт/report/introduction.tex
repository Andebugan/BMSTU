\Introduction
Словать - тип структуры данных, позволяющей хранить пары вида ключ-значение. Также словарь позволяет добавлять новые пары ключ-значение, удалять старые пары и изменять пары. Одним из основных условий работы словаря явлается уникальность ключа относительно остального множества ключей, которые на данный момент присутствуют в словаре. В этой лабораторной работе мы будем рассматривать алгоритмы поиска значения по ключу в словаре на основе словаря связей студентов с их научными руководителями. 

Целью данной лабораторной работы является изучение алгоритмов поиска значения по ключу в словаре с помощью алгоритмов полного перебора, двоичного поиска в упорядоченном словаре и алгоритме поиска с использованием частичного анализа. Для того, чтобы достичь поставленной цели, необходимо выполнить следующие задачи:

\begin{enumerate}
	\item провести анализ алгоритмов поиска в массиве;
	\item описать используемые в алгоритмах структуры данных;
	\item привести схемы рассматриваемых алгоритмов;
	\item программно реализовать описанные выше алгоритмы;
	\item провести сравнительный анализ алгоритмов относительно скорости работы.
\end{enumerate}